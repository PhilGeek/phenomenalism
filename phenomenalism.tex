%!TEX TS-program = xelatex
%!TEX TS-options = -synctex=1
%!TEX encoding = UTF-8 Unicode
%
%  mog
%
%  Created by Mark Eli Kalderon on 2010-09-22.
%  Copyright (c) 2010. All rights reserved.
%

\documentclass[12pt]{article} 

% Definitions
\newcommand\mykeywords{appearance, looks} 
\newcommand\myauthor{Mark Eli Kalderon} 
\newcommand\mytitle{Against Phenomenalism about Appearances}

% Packages
\usepackage{geometry} \geometry{a4paper} 
\usepackage{url}
\usepackage{txfonts}
\usepackage{color}
\definecolor{gray}{rgb}{0.459,0.438,0.471}
% \usepackage{setspace}
% \doublespace % Uncomment for doublespacing if necessary
\usepackage{epigraph} % optional

% XeTeX
\usepackage{fontspec}
\usepackage{xltxtra,xunicode}
\defaultfontfeatures{Scale=MatchLowercase,Mapping=tex-text}
\setmainfont{Hoefler Text}
\setsansfont{Gill Sans}
\setmonofont{Inconsolata}

% Section Formatting
\usepackage[]{titlesec}
\titleformat{\section}[hang]{\fontsize{14}{14}\scshape}{\S{\thesection}}{.5em}{}{}
\titleformat{\subsection}[hang]{\fontsize{12}{12}\scshape}{\S{\thesubsection}}{.5em}{}{}
\titleformat{\subsubsection}[hang]{\fontsize{12}{12}\scshape}{\S{\thesubsubsection}}{.5em}{}{}

% TODO List
% \usepackage{color}
% \usepackage{index} % use index package to create indices
% \newindex{todo}{tod}{tnd}{TODO List} % start todo list
% \newindex{fixme}{fix}{fnd}{FIXME List} % start fixme list
% \newcommand{\todo}[1]{\textcolor{blue}{TODO: #1}\index[todo]{#1}} % macro for todo entries
% \newcommand{\fixme}[1]{\textcolor{red}{FIXME: #1}\index[fixme]{#1}} % macro for fixme entries

% Bibliography
\usepackage[round]{natbib} 

% Title Information
\title{\mytitle} % For thanks comment this line and uncomment the line below
% \title{\mytitle\thanks{}}% 
\author{\myauthor} 
% \date{} % Leave blank for no date, comment out for most recent date

% PDF Stuff
\usepackage[plainpages=false, pdfpagelabels, bookmarksnumbered, backref, pdftitle={\mytitle}, pagebackref, pdfauthor={\myauthor}, pdfkeywords={\mykeywords}, xetex, colorlinks=true, citecolor=gray, linkcolor=gray, urlcolor=gray]{hyperref} 

%%% BEGIN DOCUMENT
\begin{document}

% Title Page
\maketitle
% \begin{abstract} % optional
% \noindent
% \end{abstract} 
\vskip 2em \hrule height 0.4pt \vskip 2em
% Main Content
\epigraph{It is perhaps even clearer that the way things look is, in general, just as much a fact about the world, just as open to public confirmation or challenge, as the way things are. I am not disclosing a fact about myself, but about petrol, when I say that petrol looks like water.---\citet[43]{Austin:1962lr}}

% Layout Settings
\setlength{\parindent}{1em}

\section{Introduction} % (fold)
\label{sec:introduction}

Like \citet{Moore:1903uo}, Cook Wilson distinguished the act of perceiving and the object of perception. In perceiving an object, the object appears to the subject, and so the subjective act of perceiving is sometimes described as an \emph{appearance}. Given the act--object distinction, an appearance, so understood, is necessarily distinguished from its object. In a reamarkable passage, however, Cook Wilson warns against a misleading ``objectification'' of appearing:
\begin{quote}
	But next the \emph{appearance}, though properly the appear\emph{ing} of the object, gets to be looked on as itself an object and the immediate object of consciousness, and being already, as we have seen, distinguished from the object and related to our subjectivity, becomes, so to say, a mere subjective `object'---`appearance' in that sense. And so, as \emph{appearance} of the object, it has now to be represented not as the object but as the phenomenon caused in our consciousness by the object. Thus for the true appearance (=appearing) to us of the \emph{object} is substituted, through the `objectification' of the appearing as appearance, the appearing to us of an \emph{appearance}, the appearing of a phenomenon caused in us by the object.  \citep[796]{Cook-Wilson:1926sf}
\end{quote}


% section introduction (end)

% Bibligography
\bibliographystyle{plainnat} 
\bibliography{Philosophy} 

\end{document}